\documentclass{article}
\usepackage{amsmath}
\usepackage{amssymb}
\usepackage{CJKutf8}
\usepackage[left=2cm, right=2cm, top=2cm, bottom=2cm]{geometry}
\usepackage{color}
\usepackage{verbatim}
\begin{document}
\begin{CJK}{UTF8}{bsmi}


\title{Generalization of least square method (廣義最小平方法) }
\date{2020.2}
\author{加減 Plus \& Minus}




\maketitle

\section{機器學習情境回顧}
給定已知 $N+1$ 筆 , $M+K$ 維"連續變數"資料集, $ \mathcal{D}:= \bigg\{\bigg({\color{red}\underbrace{x^{(n)}_1,x^{(n)}_2,...x^{(n)}_M}_{\vec{x}^{(n)}}},{\color{blue}\underbrace{y_{1}^{(n)},y_{2}^{(n)},...,y_{K}^{(n)}}_{\vec{y}^{(n)}}}\bigg)\bigg\}_{n=1}^{N+1}$
\begin{itemize}

\item $\text{切割資料集}$ $$ \forall \  \sigma \in \mathcal{S}_{N+1} , \mathcal{D} =\mathcal{D}^{\text{train}}_{\sigma} \cup \mathcal{D}^{\text{valid}}_{\sigma} ,  $$
\begin{itemize}
\item 本文只考慮 \color{red} Leave-One-Out : $|\mathcal{D}^{\text{train}}_{\sigma}| = N ,  |\mathcal{D}^{\text{valid}}_{\sigma}| = 1  $ 即 $\sigma \in S^{Loo}_{N+1} , |S^{Loo}_{N+1}| = C^{N+1}_{N} = N+1$
\end{itemize}
\item 只根據 $\mathcal{D}^{\text{train}}_{\sigma}$ , 可建構迴歸模型 $F_{\sigma} : \mathbb{R}^{M} \longrightarrow \mathbb{R}^{K} $,並同時使用 $\mathcal{D}^{\text{train}}_{\sigma}$, $\mathcal{D}^{\text{valid}}_{\sigma}$  來評估建模成效

$$\text{終極目標: } \forall \sigma \in \mathcal{S}_{N+1}, \forall n=1,2,...,N+1 \quad \underbrace{F_{\sigma}({\color{red}\vec{x}^{(n)}})}_{\text{predicted}} \approx \underbrace{{\color{blue}\vec{y}^{(n)}}}_{\text{target}} $$
\item 根據多維向量定義,可以把單一模型 $F: \mathbb{R}^{M} \longrightarrow \mathbb{R}^{K}$ 問題,想成獨立$K$個模型 $f: \mathbb{R}^{M} \longrightarrow \mathbb{R} $
$$ F_{\sigma}({\color{red}\vec{x}^{(n)}}) = \left[\begin{array}{c} f_1({\color{red}\vec{x}^{(n)}}) \\ f_2({\color{red}\vec{x}^{(n)}}) \\ ... \\ f_K ({\color{red}\vec{x}^{(n)}}) \\ \end{array} \right] \approx  \left[{\color{blue}\begin{array}{c} y_1 \\ y_2 \\... \\ y_K\\ \end{array}} \right]  = {\color{blue}\vec{y}^{(n)}} $$
\begin{itemize}
\item 於是只需研究 $f({\color{red}\vec{x}^{(n)}}) \approx {\color{blue}y^{(n)} } \in \mathbb{R}$ 迴歸問題如何建模 !!
\item 損失函數(Loss function)越小越好概念: 
$$ f({\color{red}\vec{x}^{(n)}}) \approx {\color{blue}y^{(n)} }  \Longrightarrow \bigg(f({\color{red}\vec{x}^{(n)}}) - {\color{blue}y^{(n)} }\bigg)^2 \approx 0 $$ $$ \underset{\text{overall data}}{\Longrightarrow} \underbrace{\frac{1}{N}\sum_{n=1}^{N} \bigg(f({\color{red}\vec{x}^{(n)}}) - {\color{blue}y^{(n)} }\bigg)^2}_{\text{training\_loss$(\mathcal{D}^{\text{train}}_{\sigma})$}} \approx 0  , \underbrace{\bigg(f({\color{red}\vec{x}^{(N+1)}}) - {\color{blue}y^{(N+1)} }\bigg)^2}_{\text{ validation\_loss$(\mathcal{D}^{\text{valid}}_{\sigma})$}} \approx 0  $$

\item 交叉驗證(cross validation)最大誤差越小概念:
$$\underset{\text{overall $\sigma$}}{\Longrightarrow} \underset{\sigma }{\text{max }}\text{training\_loss$(\mathcal{D}^{\text{train}}_{\sigma})$} \approx 0 , \underset{\sigma }{\text{max }}\text{validation\_loss$(\mathcal{D}^{\text{valid}}_{\sigma})$} \approx 0   $$

\end{itemize}

\end{itemize}

\newpage

\section{模型假設 = 參數化 + 線性組合假設}
\begin{itemize}
\item 
$f({\color{red}\vec{x}^{(n)}}) \overset{\text{引入模型假設}}{\Longrightarrow} \displaystyle{f({\color{blue}\vec{w}},{\color{red}\vec{x}^{(n)}}) := \sum_{b \in \mathcal{B}} {\color{blue}w_{b}} \cdot g_{b}({\color{red}\vec{x}^{(n)}}) \in Span \bigg\{g_{b} \bigg\}_{b \in \mathcal{B}} } =: Span \  g_{\mathcal{B}}\quad (\text{linear combinations of given basis $g_{\mathcal{B}}$})$

\item 常見 basis $g_{\mathcal{B}} , (M=1)$
\begin{itemize}
\item Simple Intepolation: \\$$g_{\mathcal{B}} := \bigg\{1,x_1,x_1^2,...x_1^{N-1}\bigg\}$$

\item Special Functions: \\
$$g_{\mathcal{B}} := \text{ Hermites , Chebyshevs , Legendres , Laguerres , Bessels ... }$$

\item Fourier Series: \\ $$ g_{\mathcal{B}} :=\bigg\{
e^{ikx_1}\bigg\}_{k \in \mathbb{Z}}
$$

\item ODE(微分方程) \\
$$g_{\mathcal{B}} := \bigg\{e^{\lambda x_1}\bigg\}_{\lambda \in \text{eigenvalues}}$$

\end{itemize}


\item 常見 basis $g_{\mathcal{B}} , (M\geq 1)$:
\begin{itemize}
\item Linear Regression: \\$$g_{\mathcal{B}} := \bigg\{1,x_1,x_2,...x_{M}\bigg\}$$
\item Response Surface Methodology: \\$$g_{\mathcal{B}} := \bigg\{1,\underbrace{x_1,x_2,..,x_{M}}_{\text{first-order $M$ terms }},\underbrace{x_1^2,...,x_M^2,x_1x_2,x_1x_3,...x_{M-1}x_{M}}_{\text{second-order $M^2$ terms (features interaction)}} \bigg\}$$
\item DIY or data transform by domain knowledge ...\\
\end{itemize} 

\end{itemize}

\newpage 
\section{核心推導}


\end{CJK}
\end{document}

